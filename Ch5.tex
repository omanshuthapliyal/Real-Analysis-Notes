\chapter{The Riemann-Stieljes Integral}

Suppose $f:(a,b)\to\reals$ is bounded, we want to define a quantity $\int_a^b f$ or $\int_a^b f(x) dx$ as the signed area under the graph of $f$.

In order to do so, we introduce the concept of partitions.
We say that $P=\{x_1,\cdots,x_N\}$ is a partition of $[a,b]$ if $a=x_1<x_1<\cdots<x_N=b$.
We say a partition $P$ to be ``finer'' than a partition $Q$, $P\prec Q$, if $P\supset Q$.
Clearly, refinement is a transitive property, i.e., if $P\prec Q$ and $Q\prec R$, then $P\prec R$ ($P$ finer than $Q$, $Q$ finer than $R\implies$ $P$ finer than $R$).
Also, for two partitions $P_1,P_2,\exists$ a common refinement $Q$ (think of the trivial example of $Q=P_1\cup P_2$).

\section{Riemann integral.}

\subsection{Lower and Upper sums.}
Let us define $\Delta x_i=x_{i}-x_{i-1}$, and $m_i=\inf_{[x_{i-1},x_i]} f$, $M_i=\sup{[x_{i-1},x_i]} f$.
Then we can define the lower/upper sums for function $f$, over a partition $P$ as:
\begin{equation}
\begin{split}
L(P,f) &\coloneqq \sum^N_{i=1} m_i\Delta x_i \eqqcolon \text{ lower sum}\\
L(P,f) &\coloneqq \sum^N_{i=1} M_i\Delta x_i \eqqcolon \text{ upper sum}
\end{split}
\end{equation}
This is shown for an arbitrary partition in Fig. \ref{fig:uplowSum}.
\begin{figure}[!ht]
    \centering
    \scalebox{0.80}{


% Pattern Info
 
\tikzset{
pattern size/.store in=\mcSize, 
pattern size = 5pt,
pattern thickness/.store in=\mcThickness, 
pattern thickness = 0.3pt,
pattern radius/.store in=\mcRadius, 
pattern radius = 1pt}
\makeatletter
\pgfutil@ifundefined{pgf@pattern@name@_i8bd6jylw}{
\pgfdeclarepatternformonly[\mcThickness,\mcSize]{_i8bd6jylw}
{\pgfqpoint{0pt}{0pt}}
{\pgfpoint{\mcSize+\mcThickness}{\mcSize+\mcThickness}}
{\pgfpoint{\mcSize}{\mcSize}}
{
\pgfsetcolor{\tikz@pattern@color}
\pgfsetlinewidth{\mcThickness}
\pgfpathmoveto{\pgfqpoint{0pt}{0pt}}
\pgfpathlineto{\pgfpoint{\mcSize+\mcThickness}{\mcSize+\mcThickness}}
\pgfusepath{stroke}
}}
\makeatother
\tikzset{every picture/.style={line width=0.75pt}} %set default line width to 0.75pt        

\begin{tikzpicture}[x=0.75pt,y=0.75pt,yscale=-1,xscale=1]
%uncomment if require: \path (0,478); %set diagram left start at 0, and has height of 478

%Straight Lines [id:da4315509846702619] 
\draw    (149,166) -- (149,55.5) ;


%Straight Lines [id:da30493801732215875] 
\draw    (173,165.5) -- (173,19) ;


%Straight Lines [id:da40039837814979684] 
\draw    (181,165.5) -- (181,5) ;


%Straight Lines [id:da09517758349616279] 
\draw    (65.5,166) -- (65.5,138) ;


%Straight Lines [id:da4471056983759567] 
\draw    (43.5,166) -- (43.5,149) ;


%Straight Lines [id:da7415615781500091] 
\draw    (29.5,166) -- (29.5,152) ;


%Straight Lines [id:da3290996503472099] 
\draw    (149,55) -- (172.82,54.93) ;


%Straight Lines [id:da09049690336699667] 
\draw    (43.5,149) -- (65.22,148.53) ;


%Straight Lines [id:da9546902740553911] 
\draw    (29.5,152) -- (43.62,152.13) ;


%Straight Lines [id:da6284372486571703] 
\draw [color={rgb, 255:red, 128; green, 128; blue, 128 }  ,draw opacity=1 ] [dash pattern={on 4.5pt off 4.5pt}]  (115.5,93.5) -- (30,93.5) ;


%Straight Lines [id:da86068530152694] 
\draw [color={rgb, 255:red, 128; green, 128; blue, 128 }  ,draw opacity=1 ] [dash pattern={on 4.5pt off 4.5pt}]  (149,55) -- (29,55) ;


%Straight Lines [id:da3008847504055201] 
\draw    (181,18.83) -- (173,19) ;


%Shape: Rectangle [id:dp12909102853088195] 
\draw  [pattern=_i8bd6jylw,pattern size=6pt,pattern thickness=0.75pt,pattern radius=0pt, pattern color={rgb, 255:red, 0; green, 0; blue, 0}] (115.5,93.5) -- (149,93.5) -- (149,165.5) -- (115.5,165.5) -- cycle ;
%Straight Lines [id:da7890949475160516] 
\draw    (29.5,166) -- (200,165.01) ;
\draw [shift={(202,165)}, rotate = 539.6700000000001] [color={rgb, 255:red, 0; green, 0; blue, 0 }  ][line width=0.75]    (10.93,-3.29) .. controls (6.95,-1.4) and (3.31,-0.3) .. (0,0) .. controls (3.31,0.3) and (6.95,1.4) .. (10.93,3.29)   ;

%Straight Lines [id:da542404546771029] 
\draw    (29.5,166) -- (29.01,3) ;
\draw [shift={(29,1)}, rotate = 449.83] [color={rgb, 255:red, 0; green, 0; blue, 0 }  ][line width=0.75]    (10.93,-3.29) .. controls (6.95,-1.4) and (3.31,-0.3) .. (0,0) .. controls (3.31,0.3) and (6.95,1.4) .. (10.93,3.29)   ;

%Curve Lines [id:da8638077983262378] 
\draw [color={rgb, 255:red, 208; green, 2; blue, 27 }  ,draw opacity=1 ][line width=1.5]    (29.5,152) .. controls (66,150.5) and (138.5,80) .. (181,5) ;




% Text Node
\draw (28.33,179.83) node [scale=0.8]  {$x_{1}$};
% Text Node
\draw (45.67,179.17) node [scale=0.8]  {$x_{2}$};
% Text Node
\draw (65.67,180.17) node [scale=0.8]  {$x_{3}$};
% Text Node
\draw (119,181.17) node [scale=0.8]  {$x_{i-1}$};
% Text Node
\draw (148.33,181.17) node [scale=0.8]  {$x_{i}$};
% Text Node
\draw (175.33,181.17) node [scale=0.8]  {$x_{N-1}$};
% Text Node
\draw (200.33,180.17) node [scale=0.8]  {$x_{N}$};
% Text Node
\draw (87.27,178.97) node [scale=0.8]  {$\cdots $};
% Text Node
\draw (13,93) node   {$m_{i}$};
% Text Node
\draw (13.5,55) node   {$M_{i}$};


\end{tikzpicture}
}
    \caption{Upper/lower sums for a function and a given partition}
    \label{fig:uplowSum}
\end{figure}

\subsection{Upper and Lower integrals.}
Note that every partition would have a different upper/lower sum, since if clearly depends on the partition $P$.
In order to take all partitions into account, we define the upper/lower integrals as:
\begin{equation}
\begin{split}
\loRiemannint{a}{b}f &= \sup_{P}L(P,f)\eqqcolon \text{ lower integral}\\
\upRiemannint{a}{b}f &= \inf_{P}U(P,f)\eqqcolon \text{ upper integral}
\end{split}
\end{equation}

\begin{definition}
We say a function $f$ is Riemann integrable on $[a,b]$ if $\loRiemannint{a}{b}f=\upRiemannint{a}{b}f$. 
Then we define $\int^b_a f\coloneqq \loRiemannint{a}{b}f=\upRiemannint{a}{b}f$.
\end{definition}

\section{Riemann-Stieljes Integral.}
Let $f:[a,b]\to \reals$ be bounded, and $\alpha:[a,b]\to\reals$ be an increasing function.
Let $P$ be a partition of $[a,b]$ and $\Delta\alpha_i=\alpha(x_i)-\alpha(x_{i-1})$.
Then we can define upper and lower sums \emph{with respect to} $\alpha$ as,
\begin{equation}
\begin{split}
U(P,f,\alpha)&\coloneqq\sum^N_{i=1} M_i\Delta\alpha_i\\
L(P,f,\alpha)&\coloneqq\sum^N_{i=1} m_i\Delta\alpha_i
\end{split}
\end{equation}
Similarly, we define upper/lower integrals with respect to $\alpha$ as,
\begin{equation}
\begin{split}
\loRiemannint{a}{b}f d\alpha &= \sup_{P}L(P,f,\alpha)\\
\upRiemannint{a}{b}f d\alpha&= \inf_{P}U(P,f,\alpha)
\end{split}
\end{equation}
\begin{definition}
We say a function $f$ is Riemann integrable on $[a,b]$ w.r.t. $\alpha$ if $\loRiemannint{a}{b}fd\alpha=\upRiemannint{a}{b}fd\alpha$. 
Then we define $\int^b_a f\coloneqq \loRiemannint{a}{b}fd\alpha=\upRiemannint{a}{b}fd\alpha$.
\end{definition}

\section{Riemann's Integrability Criterion.}
The refinement property of partitions allows for $L(P,f,\alpha)\leq L(P^*,f,\alpha)\leq U(P^*,f,\alpha)\leq U(P,f,\alpha)$ if $P^*\prec P$.
Note that upper and lower sums of always finer partitions would always lie in-between the upper and lower sums over other partitions.
Using this we can come up with a criterion for integrability.
\begin{theorem}
We denote a function $f$ is integrable in the sense of Riemann w.r.t. $\alpha$ as $f\in \mathcal{R}(\alpha)$.\\
If $f\in\mathcal{R}(\alpha)\iff$ for any $\varepsilon,\,\exists$ a partition $P_\varepsilon$ of $[a,b]$ such that $U(P_\varepsilon,f,\alpha)-L(P_\varepsilon,f,\alpha) = \sum^N_{i=1}(M_i-m_i)\Delta\alpha_i<\varepsilon$.
\end{theorem}

\section{Integrability theorems.}
Below are some useful properties on the integrability of functions.
\begin{enumerate}
\item If $f:[a,b]\to\reals$ is continuous $\implies f\in \mathcal{R}(\alpha)$ for any $\alpha$ increasing on $[a,b]$.
\item If $f$ is monotone and $\alpha$ continuous on $[a,b]\implies f\in \mathcal{R}(\alpha)$.
\item If $f$ is continuous everywhere on $[a,b]$ except at $x=c_1,\cdots,c_N$ and bounded on $[a,b]$, and $\alpha$ continuous at $x=c_1,\cdots,c_N\implies f\in\mathcal{R}(\alpha)$.\\
Corollary: If $f\in\mathcal{R}(\alpha)$, then modifying the function at finitely many  does not affect its integrability.
\item If $f:[a,b]\to[m,M]$ is bounded, $\alpha$ increasing, and $\varphi [m,M]\to \reals$ some continuous function, then the composition $\varphi\circ f=\varphi(f)\in\mathcal R(\alpha)$.
\item Algebraic properties of integrals:\\
If $f_1,f_2\in\mathcal{R}(\alpha)$ on $[a,b]$, then the following hold
\begin{enumerate}[label=(5\alph*)]
\item $\int^b_a f_1d\alpha+\int^b_a f_2d\alpha=\int^b_a (f_1+f_2)d\alpha$
\item $\int^b_a (cf_1)d\alpha=c\int^b_a f_1d\alpha$ for some $c\in\reals$
\item $f_1\cdot f_2\in\mathcal{R}(\alpha)$
\item If $f\in\mathcal{R}(\alpha_1),f\in\mathcal{R}(\alpha_2)$ for $\alpha_1,\alpha_2$ monotone increasing on $[a,b]$, then $f\in\mathcal{R}(\alpha_1+\alpha_2)$, i.e.,\\
$\int^b_a f d(\alpha_1+\alpha_2) = \int^b_a f d\alpha_1 \int^b_a f d\alpha_2$
\item If $f\in\mathcal{R}(\alpha)\implies f\in\mathcal{R}(c\alpha)$, where\\
$\int^b_afd(c\alpha) = c\int^b_afd\alpha$ for some $c\geq 0$
\item If $f\in\mathcal{R}(\alpha)$ on $[a,b]$, and $a<c<b$, then $f\in\mathcal{R}(\alpha)$ on $[a,c]$ and $[c,b]$, where
$\int^b_c fd\alpha = \int^c_a fd\alpha + \int^b_c fd\alpha$
\end{enumerate}
\item If $m\leq f\leq M$ on $[a,b]$ and $f\in\mathcal{R}(\alpha)\implies m[\alpha(b)-\alpha(a)]\leq \int^b_a fd\alpha\leq M[\alpha(b)- \alpha(a)]$
\item If $f\in\mathcal{R}(\alpha)\implies \abs{f}\in\mathcal{R}(\alpha)$.\\
Further, $\abs{\int^b_afd\alpha}\leq \int^b_a\abs{f} d\alpha$
\item Monotonicity of Integral.\\
If $f_1(x)\leq f_2(x)\,\forall x\in[a,b]\implies \int^b_a f_1d\alpha\leq \int^b_a f_2d\alpha$
\end{enumerate}

\subsection{Riemann-Stieljes Integral as Riemann Integral.}

We say a function $f:[a,b]\to \reals$ is \textit{differentiable} at $c\in(a,b)$ if
\begin{equation*}
f'(c)\coloneqq \lim_{x\to c}\frac{f(x)-f(c)}{x-c} = \lim_{h\to 0}\frac{f(x+h)-f(x)}{h}
\end{equation*}
A natural consequence is the \textbf{Mean-Value theorem} for continuous and differentiable functions.
Let $f$ be continuous on $[a,b]$ and follow the equation above.
Then $\exists c\in (a,b)$ such that $f'(c)=\dfrac{f(b)-f(a)}{b-a}$.

\begin{theorem}
Suppose $\alpha:[a,b]\to\reals$ is an increasing, continuous function on $[a,b]$, and differentiable on $(a,b)$.
Further assume $\alpha'\in\mathcal{R}$, then if $f:[a,b]\to\reals$ is bounded then $f\in\mathcal{R}(\alpha)\iff f\alpha'\in\mathcal R$, i.e.,
\begin{equation*}
\int^b_afd\alpha = \int^b_a f\alpha' dx
\end{equation*}
\end{theorem}

\subsection{Change of variables.}
Let $f:[a,b]\to\reals,\alpha\nearrow$ on $[a,b]$.
Suppose there is a function $\varphi:[A,B]\to [a,b]$ that is continuous, invertible, and $\varphi(A)=a,\varphi(b)=b$, then \begin{equation*}
f\circ\varphi\in\mathcal{R}(\beta),\,\beta=\alpha\circ\varphi\iff f\in\mathcal{R}(\alpha)\text{ and }\int^B_Af\circ\varphi d\beta=\int^b_afd\alpha
\end{equation*}

\begin{theorem}[Fundamental Theorem of Calculus [FTC].]
Suppose $f:[a,b]\to\reals,f\in\mathcal{R}$ and $F:[a,b]\to\reals$ is such that $F$ is continuous on $[a,b]$, and $F'(x)=f(x)$ on $(a,b)$.
Then $F$ is called the \emph{anti-derivative} of $f$, i.e., 
\begin{equation*}
    \int^b_af(x)dx=F(b)-F(a)
\end{equation*}
If $f$ is continuous on $[a,b]$, $F(x)=\int^x_afdx,\,x\in[a,b]$.
\end{theorem}

\subsubsection{Integration by parts.}
An important consequence of the FTC is the property of integration by parts.
If $f,g\in\mathcal{R}$ and $F,G$ are the respective anti-derivatives, then
\begin{equation*}
\begin{split}
\int^b_aFgdx &= F(b)G(b)-F(a)G(a)-\int^b_aGfdx\\
\equiv \int^b_a FG' dx &= FG\Big\rvert^b_a - \int^b_a Gf' dx
\end{split}
\end{equation*}
The above can be proved by applying the FTC on $FG$ so that $(FG)'=F'G + FG'\implies FG$ is the anti-derivative of $Fg+fG$.