\chapter{Constructing Reals}

Begin with the notion that we already know of the existence of natural numbers, denoted by $\naturals$ (let us skip Peano's axioms which formalize the existence of $\naturals$). Let us define the set of rational numbers $\rationals$ as $\rationals=\{\frac{m}{n}| m,n\in\naturals,n\neq 0\}$. Note that $\naturals[\rationals]$ are closed under addition and multiplication, i.e., if $a,b\in\naturals\Rightarrow a+b\in \naturals$. It can also be seen that $\naturals\in\rationals$ by definition.\\

The notion of sets is to be able to `collect' objects or elements into a `set', so to speak, so that we can perform set operations defined in Chapter 0. This provides us with a framework for constructing sets, comparing sets ($\subset, \subseteq$), and adding ($\cup$) sets together. However, the elements of the sets themselves are arbitrarily placed and have no order among them, i.e., if $A=\{a_0,a_1,a_2\}$, we can say that the element $a_1$ \textit{relates} to the set $A$ as $a_1\in A$, but we can not compare the individual elements of the set. In other words, the elements of the set so far are \textit{cardinal}, or, forming a set out of the elements is only assisting us to count them, but not \textit{order} them in any manner.\\

\section{Ordered Sets}

Consider the operation `$<$' on a set $S$ which is defined as follows:
\begin{enumerate}[label=\roman*.]
\item For some elements $x,y,z\in S$, if $x<y$ and $y<z$, then $x<z$.
\item If $x,y\in S$, then only one of the three can hold for the elements $x,y$, either $x<y$, or $x=y$ or $y<x$ (that is, $x$ either precedes, equals, or succeeds $y$).
This is called \textit{trichotomy}.\\
\textit{Note: }By trichotomy, the negation of `$<$' is `$\geq$'.
\end{enumerate}
The set $S$ together with the relation $<$ is called an \textit{ordered set}. Now that we have ordering in sets, we can talk define bounds, supremum and infimum of sets.\\

Consider a set $E\subset S$, where $S$ is an ordered set. If there exists $\beta\in S$ such that for every element $x\in E$, $x\leq \beta$, then the value $\beta$ bounds the set $E$ from above. Similarly we can define a lower bound $\beta$ for the set $E$ if $x\geq \beta$.
If the upper bound $\beta$ of $E$ satisfies the additional property that no smaller $\alpha < \beta$ exists in $S$, then $\beta$ is called the least upper bound, or $\beta = \sup{E}$. Similarly, $\alpha$ is called the infimum of $E$, $\alpha=\inf{E}$, if $\alpha$ is the greatest lower bound.\\

\begin{theorem}[Existence of supremum]
If set $E\subset S$ such that $E\neq \nullset$, and is bounded above, then $\alpha=\sup{E}$ exists in $S$. Further, if $L$ is the set of all upper bounds of $E$, then $\alpha=\inf{L}$.
\end{theorem}
The existence of supremum is essentially guaranteed for an ordered set if it is non-empty and bounded from above. In other words, if your nonempty set is bounded from above, I can always find the least upper bound. For instance, $E=\{\frac{1}{n}|n\in \naturals,n\neq0\}$. Clearly $E$ is contained in $\rationals$, which is an ordered set, and $1\in E$, so $E$ is nonempty. Further, $1$ is an upper bound for $E$, therefore, $\exists \sup{E}$. Also note that $0$ bounds $E$ from below, so $\exists \inf{E}$ as well.\\

This gives us a framework to compare elements of an ordered set, within the set, thereby forming ordinal sets. However, so far I can only tell if two elements precede, succeed, or equal each other. But we can perform other operations on these elements if we introduce the concept of \textit{fields}.

\section{Fields}
A field is a set $F$ with associated operations addition `$+$' and multiplication `$\times$' (these notes often omit the operator $\times$ to rewrite $a\times b$ as $ab$). The fields are \textit{closed} under both these operations, i.e., if $a,b\in F$, then the result of either operations $a\circ b \in F$. Axioms to define addition are:
\begin{enumerate}[label=A\arabic*.]
\item Closure: If $a,b\in F$, then $a+b\in F$
\item Commutativity: $a+b=b+a$
\item Associativity: $a+(b+c) = (a+b)+c$
\item Existence of identity: $\exists \,0\in F$ such that $x+0=x\,\forall x\in F$
\item Existence of inverse: $\exists -x\in F$ such that $x+(-x)=0\,\forall x\in F$
\end{enumerate}
Similarly, we can define multiplication using:
\begin{enumerate}[label=M\arabic*.]
\item Closure: If $a,b\in F$, then $ab\in F$
\item Commutativity: $ab=ba$
\item Associativity: $a(bc) = (ab)c$
\item Existence of identity: $\exists \,1\in F$ such that $x1=1x=x\,\forall x\in F$
\item Existence of inverse: $\exists 1/x\in F$ such that $x(1/x)=1\,\forall x\in F$
\end{enumerate}

\begin{remark}
Let us begin with the notion of $\naturals$. Now note that $\naturals$ is an ordered set. Now if we need to define an operation `$+$' that follows axioms A1-A5, we observe that $\naturals$ are incomplete. That is, if $\exists \,a\in\naturals$, then $\nexists -a\in \naturals$ such that they add up to the additive identity. In order to form a closed field under addition, we \textit{extend} $\naturals$ to $\integers$. This automatically means that $\naturals \subset \integers$, and $integers$ are closed under addition. now further notice that if we need to define a multiplication operation, $\integers$ are closed under it (i.e., M1. is satisfied), but once again $\nexists 1/x \in \integers, x\neq 0$ such that $x\cdot 1/x=1\,\forall x\in \integers$. This admits an automatic extension to the set $\mathbb{S}=\{\frac{1}{n}:n\in \integers \setminus \{0\} \}$. This forms a field $\mathbb{S}$ which satisfies M1-M5, but now does not satisfy A1 (consider $1/3, -1/4 \in\mathbb{S}$, but $1/3 + (-1/4)\notin \mathbb{S}$). This clearly demands and extension to a set $\rationals$ of the form $\naturals=\{\frac{p}{q}:p\in\integers,q\in \naturals \setminus \{0\}\}$. Thus, we get rational numbers if we start with the naturals.
\end{remark}

Note that an ordered set is not necessarily a field (example $\naturals$), and a field is not necessarily an ordered set (example $\mathbb{C}$, which will be seen later).

We can prove the following often used properties of addition and multiplication:
\begin{enumerate}[label=(a\arabic*)]
\item If $x+y=x+z$ then $y=z$ (from A3 and A4)
\item If $x+y=x$ then $y=0$ (from p1; also implies uniqueness of additive identity)
\item If $x+y=0$ then $y=-x$ (from p1 and A4; also implies uniqueness of additive inverse)
\item $-(-x)=x$ (apply A5 on $-x$)
\end{enumerate}
Very similarly,
\begin{enumerate}[label=(m\arabic*)]
\item If $x\neq0$, and $xy=yz$, then $y=z$
\item If $x\neq0$, and $xy=x$, then $y=1$
\item If $x\neq0$, and $xy=1$, then $y=1/x$
\item If $x\neq0$ then $1/(1/x)=x$
\end{enumerate}

\subsection{Ordered Fields}
If a field $F$ is an ordered field if it is also an ordered set and satisfies the following:
\begin{enumerate}[label=(OF\arabic*)]
\item $x+y<x+z$ if $x,y,z\in F$ and $y<z$
\item $xy>0$ if $x,y\in F,x>0$ and $y>0$
\end{enumerate}
A number is called \textit{positive} if $x>0$ and \textit{negative} if $x<0$.

\section{Constructing Reals}
It should be noted that the field $\rationals$ is incomplete. For example, $\exists p$ such that $p\cdot p=2$, but $p\in \rationals$. Consider Rudin's Example 1.1 which explains that if $A=\{p\in\rationals:p^2<2\}$ and $B=\{p\in\rationals:p^2>2\}$, then $A$ has no supremum in $\rationals$ and $B$ has no infimum in $\rationals$. In order to extend the least upper bound property in $\rationals$, we need to construct $\reals$.

\begin{theorem}[Real number field]
There exists an ordered field $\reals \supset \rationals$ such that $\reals$ has the least upper bound property.
\end{theorem}
\begin{proof}[Sketch of Proof]

\end{proof}

\begin{theorem}[Archimedean property]
If $x\in\reals, y\in\reals$ and $x>0$, then there exists a positive $n\in\naturals$ such that $nx>y$
\end{theorem}
\begin{corollary}
For every $x,y\in\reals$ and $x<y$, there exists some $p\in\rationals$ such that $x<p<y$.
\end{corollary}

\begin{theorem}
For every real $x>0$ and integer $n>0$ there exists a unique positive real $y$ such that $y^n=x$.
\end{theorem}
\begin{corollary}
If $a$ and $b$ are positive reals and $n\in\integers$, then $(ab)^{1/n} = a^{1/n}b^{1/n}$.
\end{corollary}

\section{Complex Field}
\begin{definition}
A complex number is an ordered pair  $(a,b)$ of real numbers $a,b$, i.e., $(a,b)\neq(b,a)$ if $a\neq b$.
\end{definition}
The set of such numbers is called $\complex$. Further defining the specific addition and multiplication for the field as:
\begin{definition}\label{complexOperators}
If $x,y\in\complex$ such that $x=(a,b),y=(c,d)$, then
\begin{equation*}
\begin{split}
x+y &= (a+c,b+d)\\
xy &= (ac-bd,ad+bc)
\end{split}    
\end{equation*}
\end{definition}

Try to prove A1-A5 and M1-M5 for the operations above in the field $\complex$.

Using these definitions, for any real number $x,y$ we have $(x,0)+(y,0)=(x+y,0)$, and $(x,0)(y,0)=(xy,0)$. This means that the multiplication and addition defined in Def. \ref{complexOperators} automatically satisfy the arithmetic of real number operations. Therefore, we constructed $\complex$ without defining $i^2=-1$, or extending $\reals$.\\
\begin{remark}
Note that we can easily prove using the definitions above that for some $i\coloneqq (0,1)$, $i\cdot i=-1$, and using that fact further prove that $(a,b)=a+bi$
\end{remark}

